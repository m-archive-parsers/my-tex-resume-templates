\documentclass{resume}\begin{document}

\pagenumbering{gobble} % suppress displaying page number

\name{\kaishu 邢健(南川)}

\contactInfo{shawninjuly@gmail.com}{(+86) 17766091857}{INTJ-T}{期望:投资经理(硬科技方向)}

\section{个人总结}

\begin{itemize}[parsep=0.2ex]

\item{十年编程经验,精通 Python、C++、TypeScript,熟练掌握常用数据结构与部分 ACM 算法,累积个人项目数十个,涉及前端、后端、算法、系统等各个领域}

\item{优秀的笔记习惯,自研基于 VsCode、Markdown 的笔记工作流,持续记录与整理个人开发笔记数百万字,部分内容公开在个人博客 markshawn.com}

\item{拥抱开源,为 Meta 的文档框架 Docusaurus、国内最火的微信机器人框架 Wechaty 等积极贡献多个 PR,曾两日内独立搭建了烂尾楼开源项目的整个数据分析框架}

\item{性格坚韧、执行力强,曾独自从成都出发,骑行三千公里历经多次生死劫难顺利抵达拉萨与珠峰大本营,目前坚持每周两次 5 公里慢跑}

\item{相信科技改变世界,致力于坦诚、连结、高效率、有价值的工作,对投资与创业充满激情,希望能与优秀的人一起成长}

\end{itemize}

\section{工作经历}

\datedsubsection{\textbf{浙江露熙科技有限公司},高级 Framework / SLAM 研发工程师}{{2021.06 - 2022.10}}

\begin{itemize}[parsep=0.2ex]

\item{影创科技鸿鹄眼镜项目:带队2人,负责与影创科技合作的鸿鹄主机的系统研发 \par 1. 从零到一在 App / Framework / Driver 层实现在鸿鹄主机上触摸同步控制眼镜内的操作系统 \par 2. 实现在鸿鹄主机上控制从眼镜读取 3DOF 数据}

\item{理想汽车VR眼镜项目:与2位同事一起,负责向理想汽车的车机移入我们的VR系统 \par 1. 独立负责 Android 7 到 11 的大版本迁移工作,重写了 OpenGL 的渲染 \par 2. 独立负责核心算法 TimeWarp 在系统内的实现,以减轻用户戴上眼镜后的眩晕感}

\item{SLAM 算法自研项目:与一位同事共同负责公司 SLAM 算法(也是自动驾驶的核心技术)的自研 \par 1. 系统梳理了2015 年到 2022 年学术界与工业界 SLAM 算法在无人机、VR/AR 眼镜上的理论研究与开源代码,并着手翻译了其中较为典型的几篇论文 \par 2. 自学高翔博士的《SLAM十四讲》,基于 OpenCV 与机器学习分别实现了可见光(60+\%准确度)、红外光(90+\%准确度)下的手柄识别,基于 ORB、VINS-Fusion 等模型对我司数据进行调优}

\end{itemize}

\datedsubsection{\textbf{安永华明会计师事务所},计算机辅助审计实习生}{{2020.01 - 2020.03}}

\begin{itemize}[parsep=0.2ex]

\item{项目开发:为实现XBRL文件的解析,使用一周时间快速掌握了XML技术,同时也查出了证监会规范中的多项错误并就这些错误与证监会进行了沟通反馈}

\item{财报审计:完成华安、 财通、 蜂巢三家基金公司百余支公募基金年审工作,保证数据准确性}

\item{SQL处理:利用 SQL 检验财务数据完整性,并使用优化主键等方式提高了数据的查询效率}

\end{itemize}

\datedsubsection{\textbf{华兴资本},FA实习生}{{2019.06 - 2019.09}}

\begin{itemize}[parsep=0.2ex]

\item{商业分析能力:与团队一起着手研究了硅谷某可计算存储公司、杭州某网红牛奶公司等多家创业公司,平均花费一周时间搜集与阅读相关资料、咨询公司高管或员工,完成对公司商业模式的分析与市场规模的预测}

\item{数据采集能力:为保证项目组投资决策的及时高效,使用 scrapy、selenium、puppeteer 等完成特定数据的采集,独立搭建了网站方便同事下载,节省了项目组搜集信息的时间(2小时→10分钟),并给项目组设计了一份爬虫教学的PPT内部学习}

\end{itemize}

\section{教育经历}

\datedsubsection{\textbf{南京审计大学},经济学学士}{{2016.09 - 2020.06}}

\begin{itemize}[parsep=0.2ex]

\item{课程:计量金融学、博弈论、宏观经济学、微观经济学、会计学}

\item{自学:数据结构与算法、计算机组成原理、操作系统、软件工程、数据库、编译原理}

\item{荣誉:中国大学生自强之星(2020)、校级十大年度人物(2020)、校级十佳青年(2020)}

\item{竞赛:华为财务精英挑战赛校冠军(2019,180+队伍)、华为软件精英挑战赛复赛第六(2019),全国大学生创青春创业比赛江苏省铜奖(2018,负责编程)}

\item{社团:校金融科技协会创始人兼首届会长、证券投资俱乐部学术指导}

\end{itemize}

\datedsubsection{\textbf{香港中文大学(深圳)},学术访问}{{2019.01 - 2019.06}}

\datedsubsection{\textbf{电子科技大学},软件工程(后退学)}{{2013.09 - 2014.10}}

\end{document}